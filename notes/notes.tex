\documentclass[10pt,a4paper]{report}
\usepackage[utf8]{inputenc}
\usepackage[english]{babel}
\usepackage{amsmath}
\usepackage{amsfonts}
\usepackage{amssymb}
\usepackage{graphicx}
\usepackage{robust-externalize}
\usepackage{xcolor}
\usepackage{algorithm}
\usepackage{algpseudocode}
\usepackage{framed}
\usepackage{titling}
\usepackage[dvipsnames]{xcolor}
\usepackage[colorlinks=true, linkcolor=blue]{hyperref}
\newcommand{\subtitle}[1]{%
  \posttitle{%
    \par\end{center}
    \begin{center}\large#1\end{center}
    \vskip0.5em}%
}
\newcommand{\yellowtriangle}[1]{%
  \mathpalette{\color{yellow}\triangle}{#1}%
}
\author{Alessio Santoro}
\title{Machine Learning and AI with Python - Notes}
\subtitle{CS109xa, Harvard}
\makeindex
\begin{document}
\maketitle
\tableofcontents
\pagebreak
\chapter{Decision Trees}
\section{Decision trees for classification}

Logistic regression is a fundamental statistical method used in machine learning for binary classification tasks. It predicts the probability of an instance belonging to a particular class.

\paragraph{Part A: Classification using trees}
You may have learned in previous courses that \textbf{logistic regression} is most effective for constructing classification boundaries when:
\begin{itemize}
	\item The classes are well-separated in the feature space
	\item The classification boundary possesses a simple geometry
\end{itemize}
The \textbf{decision boundary} is determined at the point where the probability of belonging to class 1 is equal to that of class 0.
$$P(Y=1)=1-P(Y=0)$$
$$\rightarrow P(Y=1)=0.5$$
This is equivalent to the scenario where the log-odds are zero. The log-odds are defined as:  
$$\log\left[\frac{P(Y=1)}{1-P(Y=1)}\right]=X\beta=0$$
The equation $X\beta=0$ deine an hyperplane, but can be \textbf{generalized} using higher order polynomial terms to specify a non-linear boundary hyperplane. 
\includegraphicsWeb[scale=0.6]{https://courses.edx.org/asset-v1:HarvardX+CS109xa+3T2023+type@asset+block/DT1-part-A-2.png}
In this case the blue line can be easily described as $(y=0.8x+0.1)$ or $-0.8x+y=0.1$ (assuming that, given their position as coordinates we call \textit{longitude} as $x$ and \textit{latitude} as $y$).\\\\

\includegraphicsWeb[scale=0.2]{https://courses.edx.org/asset-v1:HarvardX+CS109xa+3T2023+type@asset+block/dry_and_agricultural_circle.png}
\includegraphicsWeb[scale=0.2]{https://courses.edx.org/asset-v1:HarvardX+CS109xa+3T2023+type@asset+block/dry_and_agricultural_boxes.png}\\\\

In the first figure, we see that we can make a circular bounding box whereas in the second figure it is likely that we can make two square bounding boxes. However, as the geometric shapes in the figures become more complicated, determining the appropriate bounding boxes becomes increasingly less straightforward and more complex.\\\\

\includegraphicsWeb[scale=0.25]{https://courses.edx.org/asset-v1:HarvardX+CS109xa+3T2023+type@asset+block/DT1-part-A-4-left.png}
\includegraphicsWeb[scale=0.25]{https://courses.edx.org/asset-v1:HarvardX+CS109xa+3T2023+type@asset+block/DT1-part-A-4-right.png}\\\\

Observe that in all the datasets, the classes remain well-separated in the feature space, \textbf{yet the decision boundaries cannot be easily described using a single equation}.

While logistic regression models with linear boundaries are intuitive to interpret by examining the impact of each predictor on the log-odds of a positive classification, it is \textbf{less straightforward to interpret nonlinear decision boundaries} in the same context. 

\subsection{Interpretable Models}
People in every walk of life have long been using interpretable models for differentiating between classes of objects and phenomena.\\
Simple flow charts can be formulated as mathematical models for classification and these models have the properties we desire. Those properties are:
\begin{enumerate}
\item They possess \textbf{sufficiently complex} decision boundaries, and
\item They remain \textbf{interpretable} by humans.
\end{enumerate}
In addition, an important characteristic of these models is that their decision boundaries are locally linear. This means each segment of the decision boundary can be simply described in mathematical terms.
\paragraph{Geometry of Flow Charts}Each flow chart node corresponds to a partitioning of the feature space — specifically, the height and width of the fruit — by axis-aligned lines or (hyper)planes.
\paragraph{Prediction}Now, when presented with a data point, we can utilize its values to navigate or traverse the flow chart. This process will guide us to the model's predicted classification outcome - in other words, it leads us to land on a leaf node.

\subsection{Classification Error}
To understand the concept of classification error as a splitting criterion, consider the following figure.\\\\
\includegraphicsWeb[scale=0.6]{https://courses.edx.org/asset-v1:HarvardX+CS109xa+3T2023+type@asset+block/DT1-part-A-11.png}\\\\
Consider the region $R_2$.\\
We define Classification error as $\frac{error}{total}$,
$$=\frac{\textrm{number of minority data points}}{\textrm{total number of data points}}$$
$$=\frac{\textrm{number of majority data points}}{\textrm{total number of data points}}$$
$$=1-\frac{\textrm{number of majority data points}}{\textrm{total number of data points}}$$
In addition, when considering the region $R_2$,
$$1-\frac{\textrm{number of majority data points}}{\textrm{total number of data points}}$$
$$=1-\Psi(\triangle|R_r) = 1 - max_k(\Psi(k|R_k))$$
Where $\Psi(k|R_r)$ is the portion of training points in $R_2$ that are labeled class $k$.\\\\The table below shows how the classification error for each region is calculated.

\begin{center}
\begin{tabular}{ |c|c|c|c| } 
 \hline
  & $\bigcirc$ & $\triangle$ & Error=$1-max_k(\Psi(k|R_r))$\\\hline\hline
 $R_1$ & 6 & 0 & $1 - max(\frac{6}{6}, \frac{0}{6})= 1-1=0$\\\hline
 $R_2$ & 5 & 8 & $1 - max(\frac{5}{13}, \frac{8}{13}) = \frac{5}{13} \approx 0.38$\\\hline
\end{tabular}
\end{center}
In general: Assume we have $P$ predictors and $K$ classes. Suppose we select the $p$th predictor and split a region along the threshold $t_p$.

We can assess the quality of this split by measuring the classification error made by each newly created region by calculating:
$$\textrm{Error}(R_r|p,t_p)= 1 - max_k(\Psi(k|R_r))$$
Where $\Psi(k|R_r)$ is the proportion of training points in $R_r$ that are labeled class $k$. Here's another example:\\
\includegraphicsWeb[scale=0.6]{https://courses.edx.org/asset-v1:HarvardX+CS109xa+3T2023+type@asset+block/DT1-part-A-13.png}
\begin{center}
\begin{tabular}{ |c|c|c|c| } 
 \hline
  & $\bigcirc$ & $\triangle$ & Error $= 1-max_k(\Psi(k|R_r))$\\\hline\hline
 $R_1$ & 1 & 0 & $1 - max(\frac{1}{1}, \frac{0}{1})= 1-1=0$\\\hline
 $R_2$ & 3 & 14 & $1 - max(\frac{3}{17}, \frac{14}{17}) = \frac{3}{17} \approx 0.18$\\\hline
\end{tabular}
\end{center}
We need to calculate the \textbf{weighted average} over both regions, taking into consideration the number of points in each region, and then minimize this weighted average over the parameters $p$ and $t_p$:
$$min_{p,t_p}\left[\frac{N_1}{N}Error(R_1|p,t_p)+\frac{N_1}{N}Error(R_2|p,t_p)\right]$$
Where $N_r$ is the number of training points inside of the region $R_r$.
\subsection{Gini index}
Assume we have $P$ \textbf{predictors} and $K$ \textbf{classes}. Suppose we select the pth predictor and split a rgion along the \textbf{threshold} $t_p\in\mathbb{R}$.\\\\We can assess the quality of this split by measuring the \textbf{Gini Index} made by each newly created region by calculating:
$$Gini(R_r|p,t_p) = 1 - \sum_{k}\Psi(k|R_r)^2 $$
\includegraphicsWeb[scale=0.6	]{https://courses.edx.org/asset-v1:HarvardX+CS109xa+3T2023+type@asset+block/DT1-part-A-15.png}\\
In the image above, we see that we have a certain number of each type of point in the specified regions based on our deliminator. We can then use these point tallies to deterime the index values.\\
\begin{center}
\begin{tabular}{ |c|c|c|c| } 
 \hline
  & $\bigcirc$ & $\triangle$ & $\textrm{Gini}= 1-\sum_k\Psi(k|R_r)^2$\\\hline\hline
 $R_1$ & 6 & 0 & $1 - \left[\left(\frac{6}{6}\right)^2 + \left(\frac{0}{6}\right)^2 \right] = 1-1=0$\\\hline
 $R_2$ & 5 & 8 & $1 - \left[\left(\frac{5}{13}\right)^2 + \left(\frac{8}{13}\right)^2 \right] = \frac{80}{169} \simeq 0.47$\\\hline
\end{tabular}
\end{center}
We can now try to find the predictor $p$ and the threshold $t_p$ minimizes the weighted average Gini Index over the two regions:$$min_{p,t_p}\left[\frac{N_1}{N}Gini(R_1|p,t_p)+\frac{N_2}{N}Gini(R_1|p,t_p)\right]$$Where $N_r$ is the number of training points inside of the region $R_r$.
\subsection{Information Theory}
The last metric for evaluating the quality of a split is motivated by metrics of uncertainty in information theory.\\\\One way to quantify the strength of a signal in a particular region is to analyze the distribution of classes within the region. We compute the \textbf{entropy} of this distribution.
\paragraph{What is entropy?}In this course we are using the information theory definition. Entropy is a fundamental concept in information theory that quantifies the average amount of information or uncertainty associated with a random variable or message.\\For a random variable with a discrete distribution, the entropy is computed by:
$$H(x) = - \sum_{x\in X}\psi(x)\log_2\psi(x)$$
Entropy is therefore a value that describes how "well-mixed" or "well-separated" a data distribution is. Highly mixed distributions will have entropy near 1. Distributions with well-separated categories will have entropy near 0.
\subsection{Entropy}
Assume we have $P$ \textbf{predictors} and $K$ \textbf{classes}. Suppose we select the $k$-th predictor and split a region along the threshold $t_p\in\mathbb{R}$.\\\\
We can assess the quality of this split by measuring the \textbf{entropy of the class distribution} in each newly created region by calculating:
$$\textrm{Entropy}(R_r|p,	t_p) = - \sum_k\psi(k|R_r)\log_2\psi(k|R_r)$$
\begin{framed}
\paragraph{Note:} We are actually computing the conditional entropy of the distribution of training points amongst the $K$ classes given that the point is in region $r$.
\end{framed}
\includegraphicsWeb[scale=0.6]{https://courses.edx.org/asset-v1:HarvardX+CS109xa+3T2023+type@asset+block/DT1-part-A-19.png}
\begin{center}
\begin{tabular}{ |c|c|c|c| } 
 \hline
  & $\bigcirc$ & $\triangle$ & $\textrm{Entropy}= - \sum_k\Psi(k|R_r)\log_2\Psi(k|R_r)$\\\hline\hline
 $R_1$ & 6 & 0 & $\left[\frac{6}{6}\log_2\frac{6}{6}+\frac{0}{6}\log_2\frac{0}{6}\right] = 0$\\\hline
 $R_2$ & 5 & 8 & $\left[\frac{5}{13}\log_2\frac{5}{13}+\frac{8}{13}\log_2\frac{8}{13}\right] \approx 1.38$\\\hline
\end{tabular}
\end{center}
The entropy calculation here yields a value of $1.38$, compared to a misclassification rate of 0.38 and a Gini index of 0.47.\\\\
We can now try to find the predictor $p$ and the threshold $t_p$ minimizes the weighted average entropy over the two regions:
$$min_{p,t_p}\left[\frac{N_1}{N}Entropy(R_1|p,t_p)+\frac{N_2}{N}Entropy(R_2|p,t_p)\right]$$
Where $N_r$ is the number of training points inside of region $R_r$.
\subsection{Comparison of Criteria}
The graph below shows how our various impurity metrics compare for a fairly simple region with two classes. On the bottom is the purity $\Psi$. 
You can see that all three measures reach a maximum around purity 0.5, but they take different paths to get there. For entropy and Gini impurity, they are curved (concave downward), while the classification error is a straight line from zero at both ends of the axis to a maximum in the center. None of these are "better", they just give different weights to different values of $\Psi$.\\
\includegraphicsWeb[scale=0.5]{https://courses.edx.org/asset-v1:HarvardX+CS109xa+3T2023+type@asset+block/S1_1_purity_metrics.png}\\
Learning the "optimal" decision tree for any given set of data is \textbf{NP complete} (intractable) for numerous simple definitions of "optimal". Instead, we will use a greedy algorithm that works as follows:
\begin{algorithm}[H]
\begin{algorithmic}[1]
   \State Start with an empty decision tree (undivided feature space)
   \State Choose the "optimal" predictor on which to split, and choose the "optimal" threshold value for splitting.
	\State Recurse on each new node until stopping condition is met.
	\State For the case of classification, predict each region to have a class label based on the largest class of the training points in that region (Bayes' classifier).
\end{algorithmic}
\end{algorithm}
This is not guaranteed to give us the best decision tree (for any particular definition of "best"), but it is very likely to give us a good one.
\section{Stopping Conditions}
\subsection{Common Stopping Conditions}
The most common stopping criterion involves restricting the \textbf{maximum depth} (\texttt{max\_depth}) of the tree.\\\\
The following diagram illustrates a decision tree trained on the same dataset as the previous one. However, a \texttt{max\_depth} of 2 is employed to mitigate overfitting.\\
\includegraphicsWeb[scale=0.7]{https://courses.edx.org/asset-v1:HarvardX+CS109xa+3T2023+type@asset+block/DT1-part-B-2-depth-2.png}
Four other common simple stopping conditions are:
\begin{enumerate}
\item Do not split a region if all instances in the region \textbf{belong} to the same class.
\item Do not split a region if it would cause the number of instances in any sub-region to go below a pre-defined threshold (\texttt{min\_samples\_leaf}).
\item Do not split a region if it would cause the total number of leaves in the tree to exceed a pre-defined threshold (\texttt{max\_leaf\_nodes}).
\item Do not split if the gain is less than some pre\-defined threshold (\texttt{min\_impurity\_decrease}).
\end{enumerate}
Let's look at each one individually.
\paragraph{1. Do not split a region if all instances in the region belong to the same class.} The diagram below displays a tree where each of the end leaf nodes are clearly of the same class, therefore we can stop at this point in growing the tree further.\\
\includegraphicsWeb[scale=0.5]{https://courses.edx.org/asset-v1:HarvardX+CS109xa+3T2023+type@asset+block/DT1-part-B-3.png}\\
\paragraph{2. Do not split a region if it would cause the number of instances in any sub-region to go below a pre-defined threshold (\texttt{min\_samples\_leaf}).}In the following diagram, we set \texttt{min\_samples\_leaf} to be 4, and we can observe that we can't split the tree further because each leaf node already meets the minimum requirement.\\
\includegraphicsWeb[scale=0.3]{https://courses.edx.org/asset-v1:HarvardX+CS109xa+3T2023+type@asset+block/DT1-part-B-4.png}
\paragraph{3. Do not split a region if it would cause the total number of leaves in the tree to exceed a pre-defined threshold (\texttt{max\_leaf\_nodes}).}In the diagram below, we observe a tree with a total of 3 leaf nodes, as specified as the maximum.\\
\includegraphicsWeb[scale=0.7]{https://courses.edx.org/asset-v1:HarvardX+CS109xa+3T2023+type@asset+block/DT1-part-B-5.png}
\paragraph{4. Do not split if the gain is less than some pre-defined threshold (\texttt{min\_impurity\_decrease}).}Compute the gain in purity of splitting a region $R$ into $R_1$ and $R_2$:$$Gain(R) = \Delta(R) - m(R) - \frac{N_1}{N}M(R_1) - \frac{N_2}{N}M(R_2)$$
\begin{framed}
Where $M$ is the 	classification error, Gini index or Entropy, depending on which is chosen to use.
\end{framed}
Scikit-learn (\texttt{sklearn}) grows trees in \textbf{depth-first} fashion by defaul, but when \texttt{max\_leaf\_nodes} is specified it is grown in a best-first fashion.
\subsection{Depth-first and Best-first}
To illustrate the difference between depth-first and best-first, we’ll consider the following decision trees that predicts if a person has heart disease based on age, sex, BP and cholesterol.
\paragraph{Depth-first Growth}For depth-first, sklearn determines the best split based on impurity decrease, with the greatest amount of decrease being the best choice. Here we see a tree with \texttt{max\_depth=2}.\\
\includegraphicsWeb[scale=0.65]{https://courses.edx.org/asset-v1:HarvardX+CS109xa+3T2023+type@asset+block/DT1-part-B-6.png}
\paragraph{Best-first Growth}Here we see a series of images showing the growth of the best first tree. Each time we calculate the impurity decrease for the threshold to determine when to stop. For the first diagram we see that the impurity decrease is minimal.\\
\includegraphicsWeb[scale=0.2]{https://courses.edx.org/asset-v1:HarvardX+CS109xa+3T2023+type@asset+block/DT1-part-B-7.png}\\
In our second split, the impurity decrease is slightly larger as compared to the split from before. We want to split where the decrease is greater; therefore, we choose the second split.\\
\includegraphicsWeb[scale=0.5]{https://courses.edx.org/asset-v1:HarvardX+CS109xa+3T2023+type@asset+block/DT1-part-B-8.png}
Once we have chosen the appropriate split, we can now delibrate the next split given the remaining nodes to compare. In the diagram below we see the next nodes we will compare the splits for.\\
\includegraphicsWeb[scale=0.6]{https://courses.edx.org/asset-v1:HarvardX+CS109xa+3T2023+type@asset+block/DT1-part-B-9.png}\\\\
This process is repeated until \texttt{max\_leaf\_nodes} is reached.
	\paragraph{Depth-first vs. Best-first}The following is a comparison between depth-first or best-first growth. The image on the left shows a stopping condition for the depth. The right image has a constraint on the number of leaf nodes, which is a breadth-first stopping condition.\\
\includegraphicsWeb[scale=0.5]{https://courses.edx.org/asset-v1:HarvardX+CS109xa+3T2023+type@asset+block/DT1-part-B-10.png}
\subsection{Variance vs. Bias}
How do we decide what is the appropriate stopping condition?\\\\
\includegraphicsWeb[scale=0.47]{https://courses.edx.org/asset-v1:HarvardX+CS109xa+3T2023+type@asset+block/DT1-part-B-11.png}
Complex trees are also harder to interpret and more computationally expensive to train. This gives us a bias-variance tradeoff, like you saw in our previous course. Let us revisit these concepts in the context of decision trees.
\begin{itemize}
\item[High Bias:]Trees of low depth are not a good fit for the training data - it’s unable to capture the nonlinear boundary separating the two classes.
\item[Low Variance:]Trees of low depth are robust to slight perturbations in the training data - the square carved out by the model is stable if you move the boundary points a bit.
\item[Low Bias:] With a high depth, we can obtain a model that correctly classifies all points on the boundary (by zig-zagging around each point).
\item[High Variance:]Trees of high depth are sensitive to perturbations in the training data, especially to changes in the boundary points.
\end{itemize}
The image below gives us a visual example. On the left we have a circular area of green data points that we're trying to model with a tree of depth 4. It's not going to be very good - it's essentially fitting a round peg in a square hole. As the depth increases, we get a better fit, but you can see more and more places where the model predicts there should be green data points for no discernible reason. As we get to depth 100, the prediction is very messy, with "false positives" everywhere. We've committed the classic mistake of over-fitting our data, capturing every single point in the training set while making our future predictions worthless.\\
\includegraphicsWeb[scale=0.47]{https://courses.edx.org/asset-v1:HarvardX+CS109xa+3T2023+type@asset+block/DT1-part-B-11.png}
\subsection*{How Can We Determine the Appropriate Hyperparameters?}
\paragraph{Cross-Validation is the key} As always we rely on CV to find the optimal set of hyper-parameters.
\pagebreak
\chapter{Decision trees II}
\section{Decision trees in regression}
\subsection{Splitting Criteria}
To illustrate how we determine the splitting criteria for a regression tree, consider the following regression tree diagram:\\
\includegraphicsWeb[scale=0.2]{https://courses.edx.org/asset-v1:HarvardX+CS109xa+3T2023+type@asset+block/2_1_basic_split_graph.png}
\includegraphicsWeb[scale=0.2]{https://courses.edx.org/asset-v1:HarvardX+CS109xa+3T2023+type@asset+block/2_1_basic_split.png}
\paragraph{}Here the data has been split based on whether x is greater than or less than 6.5. Our predicted means are -0.008 and 0.697. You can tell by looking at the graph that this is not a great split.
\paragraph{}The quality of a split in the regression tree can be assessed by calculating the \textbf{mean squared error} (MSE) of the predicted outcomes for each newly generated region. We do this by subtracting the average of the observed outcomes from each observation within the region and squaring the results, then taking the mean of these squared differences:
$$\textrm{MSE}(R_r)=\frac{1}{n}\sum_{i\in R_r} (y_i-\overline{y}_{R_r})^2$$
However, this raw MSE doesn't account for the size of each region. To correct this, we also compute a weighted average of the MSEs over both regions, so that the quantity of observations in each region is factored in:
$$\textrm{MSE}(R_1,R_2) = \left[\frac{N_1}{N}\textrm{MSE}(R_1)+\frac{N_2}{N}\textrm{MSE}(R_2)\right]$$
In this formula, $N$ represents the total number of observations, while $N_1$ and $N_2	$ indicate the number of observations in each of the two regions respectively.
\paragraph{}To find the appropritate split we minimize the weighted MSE from above with respect to the predictor $p$ and the treshold $t_p$:
$$\textrm{min}_{p,t_p}\left[\frac{N_1}{N}\textrm{MSE}(R_1)+\frac{N_2}{N}\textrm{MSE}(R_2)\right]$$
Below, we present a decision tree featuring three split points to provide another example. You can see that the averages for each region are closer to the data points in that region, which means this is a better split.\\
\includegraphicsWeb[scale=0.18]{https://courses.edx.org/asset-v1:HarvardX+CS109xa+3T2023+type@asset+block/2_1_complex_split_graph.png}
\includegraphicsWeb[scale=0.18]{https://courses.edx.org/asset-v1:HarvardX+CS109xa+3T2023+type@asset+block/2_1_complex_split.png}
\subsection{Stopping Conditions}
The stopping criteria used for classification trees, such as \textbf{maximum depth} (\texttt{max\_depth}), \textbf{minimum number of samples per leaf} (\texttt{min\_samples\_leaf}), and \textbf{maximum number of leaf nodes} (\texttt{max\_leaf\_nodes}), are also applicable to regression trees in \texttt{scikit-learn}.
\paragraph{}Similar to classification trees using \textbf{purity gain}, regression trees leverage \textbf{information gain} (or \textbf{reduction in variance}) to assess the suitability of splitting a region. This metric quantifies the decrease in mean squared error (MSE) achieved by a split, guiding the tree's growth. The splitting process stops when the information gain falls below a predefined threshold.
$$\textrm{Gain}(R) = \Delta(R) = \textrm{MSE}(R)-\frac{N_1}{N}\textrm{MSE}(R_1)-\frac{N_2}{N}\textrm{MSE}(R_2)$$
\subsection{Prediction with regression trees}
In order to make a prediction for a continuous, quantitative outcome given a regression tree, we can follow these steps:
\begin{enumerate}
\item Traverse the decision tree, starting from the root and moving towards a leaf node, based on the attribute values of the data point $x_i$.
\item Once at a leaf node, the prediction is calculated as the average of the observed outcomes $y$ within that leaf node, denoted as $\overline{y}_t$. These averages are derived from the training dataset. These averages are computed from the training dataset.
\end{enumerate}
\subsection{Handling Numerical and Categorical Attributes}
Consider the following data for our discussion. An important question arises: How do we construct a decision tree with this data, which includes both numerical and categorical attributes?\\
\begin{center}
\begin{tabular}{ |c|c|c| } 
\hline
Sepal width&Color& Flower\\\hline\hline
3.0 mm & 0 & Sunflower \\\hline
3.5 mm & 1 & Rose \\\hline
4.5 mm & 2 & Orchid\\\hline
3.7 mm & 2 & Tulip\\\hline
\end{tabular}\\
\includegraphicsWeb[scale=0.15]{https://courses.edx.org/asset-v1:HarvardX+CS109xa+3T2023+type@asset+block/2_1_flower_splits_small.png}
\end{center}
As we can see, the '\textit{compare and branch}' mechanism we used in the classification tree works well for numerical attributes. However, with a categorical feature (with more than two possible values), comparisons like 'feature < threshold' are meaningless.
\paragraph{}One straightforward solution is to encode the values of a categorical feature as numerical values, treating this feature as if it were numerical. This process is known as ordinal encoding. For instance, if we encode \textbf{Yellow}=0, \textbf{Red}=1, and \textbf{Purple}=2, our decision tree could look like this:\begin{center}
\begin{tabular}{ |c|c|c| } 
\hline
Sepal width&Color& Flower\\\hline\hline
3.0 mm & {\color{yellow}0}& Sunflower \\\hline
3.5 mm & {\color{red}1}& Rose \\\hline
4.5 mm & {\color{Plum}2}& Orchid\\\hline
3.7 mm & {\color{Plum}2}& Tulip\\\hline
\end{tabular}\\
\includegraphicsWeb[scale=0.15]{https://courses.edx.org/asset-v1:HarvardX+CS109xa+3T2023+type@asset+block/2_1_flower_splits_large.png}
\end{center}
However, the specific numerical encoding of categories can affect the potential splits in the decision tree. For example if we consider the order as above the possible non-trivial splits on color are:
\begin{center}
\{\{Yellow\}, \{Red, Purple\}\} and \{\{Yellow, Red\},\{Purple\}\}
\end{center}
On the other hand, if we adopt the following different numerical encoding (Yellow = 2, Red = 0, Purple = 1):
\begin{center}
\{\{Red\}, \{Yellow, Purple\}\} and \{\{Red, Purple\}, \{Yellow\}\}
\end{center}
This shows that depending on the ordinal encoding, the splits we optimize over can be different! Nonetheless, in practice, the impact of our choice of naive encoding of categorical variables is often negligible - models resulting from \textbf{different choices of encoding will perform comparably}.
\paragraph{}What if your categorical data is not ordinal? In such cases, ordinal encoding could lead to nonsensical splits. The solution is \textbf{one-hot encoding} or \textbf{dummy encoding}. This technique, while computationally more demanding, is implemented in several computational libraries like R's \texttt{randomForest}, \texttt{H20}, and \texttt{XGBoost}.\\
\begin{center}

\begin{tabular}{ |c|c|c|c|c| } 
\hline
Sepal width&{\color{Plum}C1}&{\color{red}C2}&{\color{yellow}C3}& Flower\\\hline\hline
3.0 mm & 0&0&1& Sunflower \\\hline
3.5 mm & 0&1&0& Rose \\\hline
4.5 mm & 1&0&0& Orchid\\\hline
3.7 mm & 1&0&0& Tulip\\\hline
\end{tabular}
\end{center}
\includegraphicsWeb[scale=0.2]{https://courses.edx.org/asset-v1:HarvardX+CS109xa+3T2023+type@asset+block/2_1_flower_splits_onehot.png}\\
While decision trees in general can handle categorical data, the scikit-learn implementation currently requires converting it before use. For the most current information, always consult the \href{https://scikit-learn.org/stable/modules/tree.html#tree-algorithms}{scikit-learn documentation}.
\section{Pruning}
\subsection{Example: Master's application}
Before delving into the pruning technique, let's consider an example scenario to better understand its practical application.\\
\includegraphicsWeb[scale=0.5]{https://courses.edx.org/asset-v1:HarvardX+CS109xa+3T2023+type@asset+block/dt2-pruning-img4_tree.png}
In the given example, we have a decision tree for assessing master's applications. Each node represents a question or criterion used for evaluating applicants, such as GPA, letters of recommendation, and GRE scores. The tree's structure guides the decision-making process, ultimately leading to an admission decision.
\paragraph{}However, this tree appears quite complex, potentially indicating overfitting. Instead of halting the tree's growth prematurely, we can employ pruning techniques to obtain a simpler, yet effective, decision tree.
\subsection{Pruning techniques}
Pruning refers to the process of selectively reducing the size of a decision tree by removing branches or subtrees. Among various pruning methods, one commonly used approach is \textbf{cost complexity pruning}.
\paragraph{}The cost complexity pruning process involves assigning a cost to each node in the tree based on its complexity and accuracy. The cost is defined as follows:
$$C(T) = \textrm{Error}(T) + \alpha|T|$$
where $T$ is the Decision tree, Error($T$) is the classification error, $|T|$ is the number of leaves in the tree and $\alpha$ is a hyperparameter that balances the trade-off between tree complexity and accuracy. On this page, we're working with an example that has $\alpha=0.2$.
\paragraph{}The pruning process can be visualized as follows. Initially, we start with a full-grown decision tree, represented as $T_0$. It splits on GRE score first, then on CGPA on both sides. Below that split, it examines years of work experience and TOEFL scores. (Remember, this tree is just a mockup - it's not how the Computer Science department actually decides who gets in.)
\includegraphicsWeb[scale=0.5]{https://courses.edx.org/asset-v1:HarvardX+CS109xa+3T2023+type@asset+block/dt2-pruning-img4_tree.png}
\begin{center}
\begin{tabular}{|c|c|c|c|}
\hline
Tree & Error(T) & $|T|$ & $\textrm{Error}(T)+\alpha|T|$\\\hline
$T_0$ & 0.32 & 8 & $0.32+8\cdot0.2=1.92$\\\hline
\end{tabular}
\end{center}
We then choose subtrees, replace them with leaf nodes, and calculate the cost complexity. Below, we have one such example.\\
\includegraphicsWeb[scale=0.5]{https://courses.edx.org/asset-v1:HarvardX+CS109xa+3T2023+type@asset+block/dt2-pruning-img5_tree.png}\\
\includegraphicsWeb[scale=0.5]{https://courses.edx.org/asset-v1:HarvardX+CS109xa+3T2023+type@asset+block/dt2-pruning-img6_tree.png}
\begin{center}
\begin{tabular}{|c|c|c|c|}
\hline
Tree & Error(T) & $|T|$ & $\textrm{Error}(T)+\alpha|T|$\\\hline
$T_0$ & 0.32 & 8 & $0.32+8\cdot0.2=1.92$\\\hline
$T_1$ & 0.33 & 7 & $0.33+7\cdot0.2=1.73$\\\hline
\end{tabular}
\end{center}
In the example above we notice that the smaller tree has a slightly larger error $\textrm{Error}(T)$ but smaller complexity score $C(T)$.
\paragraph{}From there, we iteratively evaluate different pruning options to create a sequence of pruned trees $T_1, T_2, \dots, T_L$, where $T_L$ is the tree consisting of only the root node. This process follows a bottom-up approach, considering subtrees and their corresponding cost complexity scores.
\paragraph{}The pruning algorithm can be summarized as follows:
\begin{algorithm}[H]
\begin{algorithmic}[1]
   \State Start with a full tree $T_0$, where each leaf node is pure.
   \State Replace a subtree in $T_0$ with a leaf node to obtain a pruned tree $T_1$. The selection of the subtree aims to minimize the expression:\newline $$\frac{\textrm{Error}(T_1)-\textrm{Error}(T_0)}{\alpha(|T_0|-|T_1|)}$$
	\State Iterate this pruning process to obtain a sequence of pruned trees $T_0, T_1,\dots, T_L$, where $T_L$ consists of only the root node. This process follows a bottom-up approach, considering subtrees at each iteration.
	\State Select the optimal tree $T_i$ using a validation set.
\end{algorithmic}
\end{algorithm}
It's worth noting that the pruning process implicitly optimizes the cost complexity score defined earlier at each step, even though we don't explicitly compute it.
\paragraph{}To determine the optimal tree, we need to find the best value for the hyperparameter $\alpha$. This is typically achieved through cross-validation, where we assess the performance of each pruned tree using a validation set. More on this below. 
\subsection{Evaluating Pruning Options}Here we demonstrate this process in more details. Let's consider an example where we have a \textbf{full tree} $T_0$:\\
\includegraphicsWeb[scale=0.5]{https://courses.edx.org/asset-v1:HarvardX+CS109xa+3T2023+type@asset+block/dt2-pruning-img8.PNG}\\
To prune this tree, we have several possible options. We can prune individual branches or subtrees, resulting in different pruned trees denoted as $T^*$. Note that in general we do not have to prune only the last layer.\\
\includegraphicsWeb[scale=0.5]{https://courses.edx.org/asset-v1:HarvardX+CS109xa+3T2023+type@asset+block/dt2-pruning-img9.PNG}\\
For each potential pruning, we obtain a set of pruned trees.\\
\includegraphicsWeb[scale=0.5]{https://courses.edx.org/asset-v1:HarvardX+CS109xa+3T2023+type@asset+block/dt2-pruning-img10.PNG}\\
Now, the question arises: How do we determine the best pruned tree?
\paragraph{}The selection process involves maximizing the difference in the cost complexity scores between the full tree and each pruned tree. We want to find the pruned tree that maximizes $C(T)-C(T^*)$, using the same definition of $C(T)$ as we did before:
$$C(T)=\textrm{Error}(T)+\alpha|T|$$
In other words, we aim to maximize $C(T)-C(T^*)$ by comparing the error measure $E(T)$ and $E(T^*)$, as well as the tree sizes $|T|$ and $|T^*|$. Our goal is to maximize  such that:
\begin{align*}
\textrm{argmax}\left[C(T)-C(T^*)\right] 
& = \textrm{argmax}\left[E(T)-E(T^*)+\alpha|T|-\alpha|T^*|\right]\\
& = \textrm{argmax}\left[\frac{E(T)-E(T^*)}{\alpha|T^*|-\alpha|T|}-1\right]\\
& = \textrm{argmax}\left[\frac{E(T)-E(T^*)}{\alpha|T^*|-\alpha|T|}-1\right]\\
& = \textrm{argmax}\left[\frac{E(T)-E(T^*)}{\alpha|T^*|-\alpha|T|}\right]\\
& = \textrm{argmax}\left[\frac{E(T)-E(T^*)}{\alpha|T|-\alpha|T^*|}\right]\\
& = \textrm{argmax}\left[\frac{E(T^*)-E(T)}{\alpha|T|-\alpha|T^*|}\right]
\end{align*}
This iterative procedure of selecting the optimal pruned tree from a given pruned tree is repeated until we obtain the final pruned tree $T^{(L)}$. Therefore, we can instead, at each step, minimize:
$$\frac{E(T^*)-E(T)}{\alpha|T|-\alpha|T^*|}$$
This iterative pruning process considers multiple levels of pruning, leading to a sequence of pruned trees $T^{(1)},T^{(2)},\dots,T^{(L)},$.
\paragraph{}Finally, we select the optimal tree $T^{(i)}$ using validation. This tree represents the pruned version that achieves the best performance on the validation set.
\subsection{Choosing the Optimal Hyperparameter}
To find the best , we follow these steps:
\begin{algorithm}[H]
\begin{algorithmic}[1]
   \State Fix a specific value for $\alpha$.
   \State Find the best tree, $T^*$, for the given $\alpha$ by evaluating the cost complexity score $C(T)$ on the validation fold.
	\State Use cross-validation to determine the best $\alpha$ by assessing the average error measure on all validation sets.
\end{algorithmic}
\end{algorithm}
This iterative process allows us to identify the hyperparameter $\alpha$ that yields the most suitable pruned tree.
\paragraph{}By employing pruning techniques and optimizing the cost complexity score, we can effectively simplify decision trees and mitigate overfitting, resulting in more interpretable and generalized models.\pagebreak
\chapter{Bagging}
\section{Bagging}\subsection{Overfitting}
\begin{description}
\item "Underfitting" is when a model is not detailed enough to describe its data accurately. 
\item "Overfitting" is when it is too detailed, fitting the test data very accurately but giving bad predictions about future data. 
\end{description}
In the case of decision trees:
\begin{itemize}
\item When a tree is too shallow, it cannot divide the input data into enough regions. This results in a decision boundary which does not envelop the data points well. Thus, the model underfits.
\item When the tree is too deep it cuts the input space into too many regions and fit to the noise of the data so it overfits. Look at the data set below, with tree depths of 1, 2, 4, 20, 36, and 100.
\end{itemize}
\includegraphicsWeb[scale=1]{https://courses.edx.org/asset-v1:HarvardX+CS109xa+3T2023+type@asset+block/3_1_boundary_1.png}
\includegraphicsWeb[scale=1]{https://courses.edx.org/asset-v1:HarvardX+CS109xa+3T2023+type@asset+block/3_1_boundary_2.png}
\includegraphicsWeb[scale=1]{https://courses.edx.org/asset-v1:HarvardX+CS109xa+3T2023+type@asset+block/3_1_boundary_4.png}\\
\includegraphicsWeb[scale=1]{https://courses.edx.org/asset-v1:HarvardX+CS109xa+3T2023+type@asset+block/3_1_boundary_20.png}
\includegraphicsWeb[scale=1]{https://courses.edx.org/asset-v1:HarvardX+CS109xa+3T2023+type@asset+block/3_1_boundary_36.png}
\includegraphicsWeb[scale=1]{https://courses.edx.org/asset-v1:HarvardX+CS109xa+3T2023+type@asset+block/3_1_boundary_100.png}
\paragraph{}You can see that the trees of depth 36 and depth 100 are near-perfect fits, but their predictions for future data are probably nonsense. \textit{Data has noise in it} - there's no point in trying to draw such a complicated boundary when a simpler one will be easier to use and interpret, and just as accurate.
\paragraph{}Here's another view. The following figure shows three plots of increasing decision boundary complexity, from left to right.\\
\includegraphicsWeb[scale=0.5]{https://courses.edx.org/asset-v1:HarvardX+CS109xa+3T2023+type@asset+block@3_1_boundary_Tree_Depth_Squares.png}
\paragraph{}And another example: in the plots below, regardless of increase in tree depth, the decision boundaries and accuracy scores are limited.
\paragraph{}\includegraphicsWeb[scale=0.35]{https://courses.edx.org/asset-v1:HarvardX+CS109xa+3T2023+type@asset+block/bagging_md_img17_d2.png}
\includegraphicsWeb[scale=0.35]{https://courses.edx.org/asset-v1:HarvardX+CS109xa+3T2023+type@asset+block/bagging_md_img17_d5.png}
\includegraphicsWeb[scale=0.35]{https://courses.edx.org/asset-v1:HarvardX+CS109xa+3T2023+type@asset+block/bagging_md_img17_d100.png}
\paragraph{}The plot on the right displays depth vs accuracy score for these kinds of classification. It shows that as the depth of the tree increases, the gap between the train and test scores increases.
\begin{center}
\includegraphicsWeb[scale=0.5]{https://courses.edx.org/asset-v1:HarvardX+CS109xa+3T2023+type@asset+block/bagging_md_img18.png}
\end{center}
There are two techniques you can use to avoid overfitting:
\begin{itemize}
\item Pruning the tree
\item Limiting the maximum depth of the tree
\end{itemize}
\subsection{Ensemble Learning}
Ensemble learning is a machine learning technique that combines several base models in order to produce one optimal predictive model.
\begin{itemize}
\item For classification, we return a \textit{plurality} of the models.
\item For regression, we return the \textit{average} of the output of the model.
\end{itemize}
A real-world example of ensemble learning is shown in the figure below. Doctors are trained on varying samples of MRI in their respective Med school. In the hospital they are currently working at, they are shown an MRI that none have seen before. Each one of them makes a decision on the presence of brain tumor. The final decision is an aggregation of all their votes.\\
\includegraphicsWeb[scale=0.75]{https://courses.edx.org/asset-v1:HarvardX+CS109xa+3T2023+type@asset+block/3_1_brain_tumor_training.png}
\paragraph{Advantages}
\begin{itemize}
\item A group of models gives a better accuracy as compared to the individual models
\item Using a single model causes overfitting, whereas the group is less likely to overfit.
\item It helps reduce bias and variance
\end{itemize}
\paragraph{Types of Ensemble methods}Kinds of ensemble methods include...
\begin{itemize}
\item Bagging \item Boosting \item Stacking \item Blending
\end{itemize}
Bagging and boosting are going to be our main focus as we move forward in this course.
\subsection{Bootstrapping}
\paragraph{Motivation:}In practice, we do not have different or unlimited datasets to train various models. However, we still want multiple models trained on different datasets. Bootstrapping is the way we accomplish this.
\end{document}